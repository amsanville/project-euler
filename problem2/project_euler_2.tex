\documentclass{article}
\usepackage{mathtools}
\usepackage{amsmath, amssymb}
\begin{document}
\section*{Problem Statement}
Each new term in the Fibonacci sequence is generated by adding the previous two terms. By starting with 1 and 2, the first 10 terms will be:
\begin{gather*}
	1, 2, 3, 5, 8, 13, 21, 34, 55, 89, ...
\end{gather*}
\noindent By considering the terms in the Fibonacci sequence whose values do not exceed four million, find the sum of the even-valued terms.
\section*{Solution}
Note Fibonacci numbers are defined by the following recursive formula:
\begin{gather*}
	a_{n} = a_{n - 1} + a_{n - 2}\\
	a_{0} = 0\\
	a_{1} = 1
\end{gather*}
where the problem statement starts with what I am calling $a_{2}$.

From below, an explicit formula for the Fibonacci numbers is:
\begin{gather*}
	a_{n} = \frac{\sqrt{5}}{5}\left(\frac{1 + \sqrt{5}}{2}\right)^{n} -\frac{\sqrt{5}}{5}\left(\frac{1 - \sqrt{5}}{2}\right)^{n} < \frac{\sqrt{5}}{5}\left(\frac{1 + \sqrt{5}}{2}\right)^{n}
\end{gather*}
Thus, I first find $n$ such that:
\begin{gather*}
	\frac{\sqrt{5}}{5}\left(\frac{1 + \sqrt{5}}{2}\right)^{n} < 4 \times 10^{6}\\
	n\ln\left(\frac{1 + \sqrt{5}}{2}\right) < \ln\left(4\sqrt{5} \times 10^{6}\right)\\
	n < \frac{\ln\left(4\sqrt{5} \times 10^{6}\right)}{\ln\left(\frac{1 + \sqrt{5}}{2}\right)} \approx 33.26
\end{gather*}
So, putting $n = 33$ and $n = 34$ into the equation:
\begin{gather*}
	a_{33} = 3,524,578\\
	a_{34} = 5,702,887
\end{gather*}
So $n = 33$ is the largest of the Fibonacci numbers to consider. Finally, due to the parity of even and odd numbers, we see that every third Fibonacci number is even. Even number plus even number is even, odd number plus odd number is even, but other combinations are odd. The sequence starts with an even and an odd number, so $a_{2} = a_{1} + a_{0}$ is odd, $a_{3} = a_{2} + a_{1}$ is even, $a_{4} = a_{3} + a_{2}$ is odd, and so on. Thus summing only the even Fibonacci numbers below 4 million is equivalent to:
\begin{gather*}
	S = \sum_{k = 0}^{11}a_{3k} = \frac{\sqrt{5}}{5}\sum_{k = 0}^{11}\left[\left(\left(\frac{1 + \sqrt{5}}{2}\right)^{3}\right)^{k} - \left(\left(\frac{1 - \sqrt{5}}{2}\right)^{3}\right)^{k}\right]
\end{gather*}
Finally, using:
\begin{gather*}
	\sum_{k = 0}^{n}c^{k} = \frac{1 - c^{n + 1}}{1 - c}
\end{gather*}
In this case, $n = 11$ and $c = (1 \pm \sqrt{5})^{3}/8$:
\begin{gather*}
	S = \frac{\sqrt{5}}{5}\left(\frac{1 - \left(\frac{1 + \sqrt{5}}{2}\right)^{36}}{1 - \left(\frac{1 + \sqrt{5}}{2}\right)^{3}} - \frac{1 - \left(\frac{1 - \sqrt{5}}{2}\right)^{36}}{1 - \left(\frac{1 - \sqrt{5}}{2}\right)^{3}}\right) = 4,613,732
\end{gather*}
\section*{Solving Linear Homogeneous Recursive Systems}
For reference, much of the information here can found at http://nms.lu.lv/wp-content/uploads/2016/04/21-linear-recurrences.pdf. The Fibonacci numbers fall under a more general class of problems called linear homogeneous recursive systems. In these problems, the goal is to convert a specific recursive formula into an explicit formula. The solutions work much like linear homogeneous ordinary differential equations. Once a solution or collection of solutions is found, we can take linear combinations of those solutions to solve the system. To see this, let me first present the system of interest: consider, a sequence of numbers $a_{n}$ such that:
\begin{gather*}
	a_{n} = c_{1}a_{n - 1} + c_{2}a_{n - 2} + ... + c_{k}a_{n - k}
\end{gather*}
where $c_{i}$ are not all uniformly zero and some initial condition $a_{i}$ for $0 \le i < k$ provided. Such a system is called a linear homogeneous recursive system with $k$ degrees of freedom. Observe that if we have one sequence of numbers $a_{i}$ that satisfy the equation and a second sequence of numbers $b_{i}$ that satisfy the system then:
\begin{gather*}
	d_{i} = \alpha_{a}a_{i} + \alpha_{b}b_{i}
\end{gather*}
also satisfies the system. In other words, linear combinations of sequences also satisfy the recursive relationship. To see this, just plug in the solution and expand as follows:
\begin{align*}
	d_{n} &= \alpha_{a}a_{n} + \alpha_{b}b_{n}\\
	&= \alpha_{a}\left(c_{1}a_{n - 1} + c_{2}a_{n - 2} + ... + c_{k}a_{n - k}\right) + \alpha_{b}\left(c_{1}b_{n - 1} + c_{2}b_{n - 2} + ... + c_{k}b_{n - k}\right)\\
	&= c_{1}\left(\alpha_{a}a_{n - 1} + \alpha_{b}b_{n - 1}\right) + ... + c_{k}\left(\alpha_{a}a_{n - k} + \alpha_{b}b_{n - k}\right)\\
	&= c_{1}d_{n - 1} + ... + c_{k}d_{n - k}
\end{align*}
Thus the $d_{n}$ satisfy the relationship. Since the recursive system has $k$ degrees of freedom from the $k$ initial conditions, I expect that if we find $k$ linearly independent sequences $s_{ij}$ (here I have the $i$th term in the $j$th such sequence), that I can solve for $a_{n}$ based on the $a_{k}$ by solving the linear system:
\begin{gather*}
	a_{i} = \sum_{j = 0}^{k - 1}\alpha_{j}s_{ij}
\end{gather*}
for the $0 \le i < k$. If these $s_{ij}$ are actually linearly independent, this will create a solve-able system of linear equations for the $\alpha_{j}$ and then:
\begin{gather*}
	a_{n} = \sum_{j = 0}^{k - 1}\alpha_{j}s_{nj}
\end{gather*}
In order for this to actually be useful, we then also want the $n$ term of each of our sequences to be readily computable. Note, this is not a formal proof of this being the solution, like for ODEs, but it does work at least in some cases.

Fortunately, such systems have a so-called characteristic polynomial that solves the system for a general term $a_{n}$ with a readily computable formula. This polynomial is:
\begin{gather*}
	P(r) = r^{k} - c_{1}r^{k - 1} - c_{2}r_{k - 2} - ... - c_{k}
\end{gather*}
The roots of this polynomial then make up the explicit formula for our solution. Observe:
\begin{gather*}
	P(r) = 0 \implies r^{k} - c_{1}r^{k - 1} - c_{2}r_{k - 2} - ... - c_{k} = 0\\
	r^{n-k}\left(r^{k} - c_{1}r^{k - 1} - c_{2}r^{k - 2} - ... - c_{k}\right) = 0\\
	r^{n} - c_{1}r^{n - 1} - c_{2}r^{n - 2} - ... - c_{k}r^{n - k} = 0\\
	r^{n} = c_{1}r^{n - 1} + c_{2}r^{n - 2} + ... + c_{k}r^{n - k}
\end{gather*}
so if $a_{n} = r^{n}$ then:
\begin{gather*}
	a_{n} = c_{1}a_{n - 1} + c_{2}a_{n - 2} + ... + c_{k}a_{n - k}
\end{gather*}
which is exactly the recursive formula. By taking linear combinations of these root solutions we then can get the particular solution for the initial condition specified by the $a_{i}$, $0 \le i < k$. 

As an example, consider the Fibonacci sequence. Note that if I call:
\begin{gather*}
	a_{0} = 0\\
	a_{1} = 1\\
\end{gather*}
Then my $a_{2}$ is the first term in the sequence specified in the problem statement. Recall that the Fibonacci numbers are defined as:
\begin{gather*}
	a_{n} = a_{n - 1} + a_{n - 2}
\end{gather*}
Thus the characteristic polynomial is:
\begin{gather*}
	r^{2} - r - 1 = 0
\end{gather*}
which by quadratic formula is:
\begin{gather*}
	r = \frac{1 \pm \sqrt{1 + 4}}{2} = \frac{1 \pm \sqrt{5}}{2}\\
	r_{1} = \frac{1 + \sqrt{5}}{2}\\
	r_{2} = \frac{1 - \sqrt{5}}{2}
\end{gather*}
Using a linear combination of these roots:
\begin{gather*}
	a_{n} = \alpha_{1}r^{n} + \alpha_{2}r^{n}\\
	a_{0} = \alpha_{1} + \alpha_{2}\\
	a_{1} = \alpha_{1}r_{1} + \alpha_{2}r_{2}
\end{gather*}
Since, $a_{0} = 0$ and $a_{1} = 1$
\begin{gather*}
	\alpha_{1} + \alpha_{2} = 0 \implies \alpha_{2} = -\alpha_{1}\\
	\left(\frac{1+\sqrt{5}}{2}\right)\alpha_{1} + \left(\frac{1 - \sqrt{5}}{2}\right)\alpha_{2} = 1\\
	\left(\left(\frac{1+\sqrt{5}}{2}\right) -\left(\frac{1 - \sqrt{5}}{2}\right)\right)\alpha_{1} = 1\\
	\sqrt{5}\alpha_{1} = 1\\
	\alpha_{1} = \frac{\sqrt{5}}{5}\\
	\alpha_{2} = -\frac{\sqrt{5}}{5}
\end{gather*}
So:
\begin{gather*}
	a_{n} = \frac{\sqrt{5}}{5}\left(\frac{1 + \sqrt{5}}{2}\right)^{n} -\frac{\sqrt{5}}{5}\left(\frac{1 - \sqrt{5}}{2}\right)^{n} 
\end{gather*}

One thing not mentioned in the above is recursive formulas with characteristic polynomials with roots of higher multiplicity. Like with ODEs, it turns out that:
\begin{gather*}
	a_{n} = nr^{n}, n^{2}r^{n}, ..., n^{m}r^{n}
\end{gather*}
where $m$ is the multiplicity are also solutions. Observe:
\begin{align*}
	a_{n} = nr^{n} &\iff nr^{n} = c_{1}(n - 1)r^{n - 1} + c_{2}(n - 2)r^{n - 2} + ... + c_{k}(n - k)r^{n - k}\\
	a_{n} = nr^{n} &\iff r^{n - k}\left(nr^{k} - c_{1}(n - 1)r^{k - 1} + c_{2}(n - 2)r^{k - 2} + ... + c_{k}(n - k)\right) = 0\\
	a_{n} = nr^{n} &\iff r^{n - k}\left((n - k)\left(r^{k} - c_{1}r^{k - 1} + c_{2}r^{k - 2} + ... + c_{k}\right)\right.\\
	&\quad+ r\left.\left(kr^{k - 1} - c_{1}(k - 1)r^{k - 2} + c_{2}(k - 2)r^{k - 3} + ... + c_{k - 1}\right)\right) = 0
\end{align*}
where the above splitting and factorization gives the characteristic polynomial and it's derivative. Observe:
\begin{align*}
	P(r) &= r^{k} - c_{1}r^{k - 1} - c_{2}r^{k - 2} - ... - c_{k}\\
	P'(r) &= kr^{k - 1} - c_{1}(k - 1)r^{k - 2} - c_{2}(k - 2)r^{k - 3} - ... - c_{k - 1}
\end{align*}
So the above becomes:
\begin{gather*}
	nr^{n - k}((n - k)P(r) + rP'(r)) = 0
\end{gather*}
So if both $P(r) = 0$ and $P'(r) = 0$, then the above is true. However, we've specifically stated that $r$ is a root of at least multiplicity 2, so:
\begin{gather*}
	P(r) = 0\\
	P(x) = (x - r)^{m}\hat{P}(x)
\end{gather*}
where $\hat{P}(x)$ is some other polynomial and $m \ge 2$, but then:
\begin{gather*}
	P'(x) = m(x - r)^{m - 1}\hat{P}(x) + (x - r)^{m}\hat{P}'(x)
\end{gather*}
and we see that $P'(r) = 0$. Thus, we conclude that $nr^{n}$ is also a solution of recursive formula if the root has a multiplicity greater than or equal to 2. Note, for the higher order terms are similar, except instead of stopping at $P'$, the formula requires higher order derivatives. The steps for the proof are as follows:
\begin{itemize}
 	\item Plug in the proposed solution to the recursive formula.
 	\item Show that the formula holds if $r$ is actually a root of the polynomial and the appropriate derivatives.
 	\item Show that because the root has the appropriate multiplicity it is actually a root of the characteristic polynomial and it's derivatives.
 	\item Conclude that $n^{m}r^{n}$ is a sequence that obeys the recursive relationship.
\end{itemize}
\subsection*{Fibonacci Numbers and the Golden Ratio}
Two numbers are said to be in the golden ratio if $a > b > 0$ and:
\begin{gather*}
	\frac{a + b}{a} = \frac{a}{b}
\end{gather*}
Or rearranging:
\begin{gather*}
	a^{2} - ab - b^{2} = 0
\end{gather*}
Thus, choosing $b = 1$, we have:
\begin{gather*}
	a^{2} - a - 1 = 0
\end{gather*}
This is the same as the characteristic polynomial for the Fibonacci numbers. And since $a > b$, we have:
\begin{gather*}
	a = \frac{1 + \sqrt{5}}{2}
\end{gather*}
And since $b = 1$, we see:
\begin{gather*}
	a = \frac{a}{b} = \frac{1 + \sqrt{5}}{2}
\end{gather*}
which is called the golden ratio. Recall that:
\begin{gather*}
	a_{n} = \frac{\sqrt{5}}{5}\left(\frac{1 + \sqrt{5}}{2}\right)^{n} -\frac{\sqrt{5}}{5}\left(\frac{1 - \sqrt{5}}{2}\right)^{n} 
\end{gather*}
for the Fibonacci numbers and observe, that:
\begin{gather*}
	\frac{1 - \sqrt{5}}{2} < 1
\end{gather*}
So in the limit that $n \rightarrow \infty$:
\begin{gather*}
	\lim_{n \rightarrow \infty} \frac{\sqrt{5}}{5}\left(\frac{1 - \sqrt{5}}{2}\right)^{n} = 0
\end{gather*}
So the ratio:
\begin{gather*}
	\lim_{n \rightarrow \infty}\frac{a_{n}}{a_{n - 1}} = \lim_{n \rightarrow \infty} \frac{\frac{\sqrt{5}}{5}\left(\frac{1 + \sqrt{5}}{2}\right)^{n} -\frac{\sqrt{5}}{5}\left(\frac{1 - \sqrt{5}}{2}\right)^{n}}{\frac{\sqrt{5}}{5}\left(\frac{1 + \sqrt{5}}{2}\right)^{n - 1} -\frac{\sqrt{5}}{5}\left(\frac{1 - \sqrt{5}}{2}\right)^{n - 1}} = \frac{\left(\frac{1 + \sqrt{5}}{2}\right)^{n}}{\left(\frac{1 + \sqrt{5}}{2}\right)^{n - 1}} = \frac{1 + \sqrt{5}}{2} 
\end{gather*}
Or exactly the golden ratio.
\subsection*{An Alternative Formula for $n$th Fibonacci Number}
In this section, I derive an alternative, but not necessarily more useful formula for the $n$th Fibonacci number using binomial expansion. In the above, we derived a formula for the $n$th Fibonacci number:
\begin{gather*}
	a_{n} = \frac{\sqrt{5}}{5}\left(\frac{1 + \sqrt{5}}{2}\right)^{n} -\frac{\sqrt{5}}{5}\left(\frac{1 - \sqrt{5}}{2}\right)^{n}\\
	a_{n} = \frac{\sqrt{5}}{5}\frac{1}{2^{n}}\left(\left(1 + \sqrt{5}\right)^{n} - \left(1 - \sqrt{5}\right)^{n}\right) 
\end{gather*}
The goal here is to simplify this formula using the binomial expansion:
\begin{gather*}
	(x + y)^{n} = \sum_{k = 0}^{n}{n \choose k}x^{n - k}y^{k}\\
	(x - y)^{n} = \sum_{k = 0}^{n}{n \choose k}x^{n - k}(-y)^{k}
\end{gather*}
Note, these formula's are simply a matter of counting all the combinations, which is exactly what $n$ choose $k$ tells us: choose $k$ different $y$'s out $n$ different bins. So:
\begin{gather*}
	(x + y)^{n} - (x - y)^{n} = \sum_{k = 0}^{n}{n \choose k}x^{n - k}\left(y^{k} - (-y)^{k}\right) = \sum_{k = 0}^{n}{n \choose k}x^{n - k}y^{k}\left(1 + (-1)^{k + 1}\right)
\end{gather*}
which only keeps the odd values of $k$:
\begin{gather*}
	\sum_{k = 0}^{n}{n \choose k}x^{n - k}y^{k}\left(1 + (-1)^{k + 1}\right) = 2\left({n \choose 1}x^{n - 1}y^{1} + {n \choose 3}x^{n - 3}y^{3} + ... + {n \choose n - 1}x^{1}y^{n - 1}\right)
\end{gather*}
(if $n$ is even otherwise the last term will be $k = n$ not $k = n - 1$). Since $x = 1$ and $y = \sqrt{5}$ in this case:
\begin{gather*}
	\left(1 + \sqrt{5}\right)^{n} - \left(1 - \sqrt{5}\right)^{n} = \sum_{k = 0}^{n}{n \choose k}\sqrt{5}^{k}\left(1 + (-1)^{k + 1}\right) = 2\sqrt{5}\sum_{k = 1}^{n}{n \choose k}\sqrt{5}^{k - 1}\left(\frac{1 + (-1)^{k + 1}}{2}\right)
\end{gather*}
And:
\begin{gather*}
	a_{n} = \frac{1}{2^{n - 1}}\sum_{k = 1}^{n}{n \choose k}\sqrt{5}^{k - 1}\left(\frac{1 + (-1)^{k + 1}}{2}\right)
\end{gather*}
where:
\begin{gather*}
\frac{1 + (-1)^{k + 1}}{2} = \begin{cases}
	0,\text{ }k\text{ even}\\
	1,\text{ }k\text{ odd}
\end{cases}
\end{gather*}
Finally, keeping only the odd values for $k$, we see that:
\begin{gather}
	k - 1 = 2m
\end{gather}
where $m$ is an integer, so:
\begin{gather*}
	\sqrt{5}^{k - 1} = 5^{m}
\end{gather*}
and
\begin{gather*}
	a_{n} = \begin{dcases*}
		\frac{1}{2^{n - 1}}\sum_{m = 0}^{n/2 - 1}{n \choose 2m + 1}5^{m},\text{ }n\text{ even}\\
		\frac{1}{2^{n - 1}}\sum_{m = 0}^{(n - 1)/2}{n \choose 2m + 1}5^{m},\text{ }n\text{ odd}
	\end{dcases*}
\end{gather*}
\section*{Geometric Sums}
Another result in this solution is the geometric sum:
\begin{gather*}
	\sum_{k = 0}^{n}c^{k} = \frac{1 - c^{n + 1}}{1 - c}
\end{gather*}
For proof consider:
\begin{gather*}
	S_{n} = \sum_{k = 0}^{n}c^{k}\\
	S_{n + 1} = \sum_{k = 0}^{n + 1}c^{k}
\end{gather*}
Then:
\begin{gather*}
	S_{n + 1} - S_{n} = c^{n + 1}\\
	S_{n + 1} = 1 + cS_{n} \implies c^{n + 1} = 1 + cS_{n} - S_{n}\\
	S_{n} = \frac{c^{n + 1} - 1}{c - 1}\\
	S_{n} = \frac{1 - c^{n + 1}}{1 - c}
\end{gather*}
\end{document}