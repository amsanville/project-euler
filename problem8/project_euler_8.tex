\documentclass{article}
\usepackage{mathtools}
\usepackage{amsmath, amssymb}
\begin{document}
\section*{Problem Statement}
The four adjacent digits in the 1000-digit number that have the greatest product are $9 \times 9 \times 8 \times 9 = 5832$.\\

\begin{center}
73167176531330624919225119674426574742355349194934\\
96983520312774506326239578318016984801869478851843\\
85861560789112949495459501737958331952853208805511\\
12540698747158523863050715693290963295227443043557\\
66896648950445244523161731856403098711121722383113\\
62229893423380308135336276614282806444486645238749\\
30358907296290491560440772390713810515859307960866\\
70172427121883998797908792274921901699720888093776\\
65727333001053367881220235421809751254540594752243\\
52584907711670556013604839586446706324415722155397\\
53697817977846174064955149290862569321978468622482\\
83972241375657056057490261407972968652414535100474\\
82166370484403199890008895243450658541227588666881\\
16427171479924442928230863465674813919123162824586\\
17866458359124566529476545682848912883142607690042\\
24219022671055626321111109370544217506941658960408\\
07198403850962455444362981230987879927244284909188\\
84580156166097919133875499200524063689912560717606\\
05886116467109405077541002256983155200055935729725\\
71636269561882670428252483600823257530420752963450
\end{center}


\noindent Find the thirteen adjacent digits in the 1000-digit number that have the greatest product. What is the value of this product?
\section*{Solution}
Honestly, overall not a very interesting problem. There are two complications in this problem: the number and product might exceed the maximum number of digits the data types can hold and divide by zero errors. Nominally, though, I would implement a class that would handle the representation as well as basic arithmetic. Python addresses the first one with it's own internal way of handling integer representation. However, for fun, I implemented a version of this code in python. Note, in languages like FORTRAN, you can create data types of arbitrary size, also avoiding this problem. I'm not completely up-to-date on what languages like Java or C++ can do, but there is likely already a library that exists to solve this problem.

As for the second complication, the naive algorithm of multiplying by the next number and dividing by the previous number to create the product breaks due to the inclusion of zero. In the code, each product is calculated from scratch at each index giving an order $nk$ algorithm, where $n$ is the number of adjacent digits of interest and $k$ is the number of digits in the number. In this case $n = 13$ and $k = 1000$. 

The solution then is:
\begin{gather}
	5 \times 5 \times 7 \times 6 \times 6 \times 8 \times 9 \times 6 \times 6 \times 4 \times 8 \times 9 \times 5 = 23514624000
\end{gather}
\end{document}